\section{Sistemi DEA (Distribuite, Eterogenee e Autonome)}
In un sistema distribuito si hanno tante basi di dati locali, diverse applicazioni su ogni nodo di elaborazione (architetture shared nothing).  
Esempio: passaggio, in una singola organizzazione, da base dati centralizzata a distribuita. Unica base di dati, localizzata a Milano acceduta da tutti i nodi con transazioni centralizzate. Può essere utile per esigenze di efficienza o affidabilità spostare o duplicare alcuni dati presso altri nodi frammentando la base di dati in frammenti, scelti ad esempio sulla base degli accessi dai vari nodi. 
\begin{table}[h!]\centering
\begin{tabular}{|l|l|l|l|}
\hline
Sistema               & Distribuzione & Eterogeneità & Autonomia \\ \hline
Base dati distribuite & Alta          & Bassa        & Bassa     \\ \hline
Multidatabase         & Alta          & Alta         & Alta      \\ \hline
\end{tabular}
\end{table}

Nella realizzazione di un sistema con caratteristiche di distribuzione si pone il problema di cosa distribuire.
Un sistema distribuito è un sistema per cui si verifica almeno una delle seguenti condizioni:
\begin{itemize}
    \item Le applicazioni, fra loro cooperanti, risiedono su più nodi elaborativi (elaborazione distribuita)
    \item L’archivio informativo è distribuito su più nodi (base di dati distribuita)
\end{itemize}
Ad esempio la workstation ed un server possono ripartirsi le funzionalità di elaborazione, oppure su una schiera di sistemi server possono essere ripartiti diversi sottoprogrammi che implementano diverse funzionalità.

La distribuzione si dice essere \textit{ortogonale e trasparente} agli altri.\\
È importante in fase di definizione architetturale capire cosa distribuire. Capire cosa distribuire è una parte consistente dello studio di come costruire un'architettura distribuita (magari frutto di situazioni particolari come la fusione di due sistemi a causa di un'acquisizione aziendale etc, dove diverse logiche applicative e diverse strutture dati possono creare situazioni molto pericolose).\\

\subsection{Criteri di classificazione dei database distribuiti}
Un \textbf{DBMS Distribuito Eterogeneo Autonomo} è in generale una federazione di DBMS che collaborano nel fornire servizi di accesso ai dati con livelli di \textit{trasparenza} definiti (infatti le diversità tra db nei nodi vengono nascosti a vari \textit{livelli di trasparenza} per distribuzione, eterogeneità e autonomia). 
L’esigenza di integrare a posteriori (o talvolta progettare fin dall’inizio) sistemi di tipo federato emerge in molti casi:
\begin{itemize}
    \item Evoluzione e/o integrazione di componenti applicativi sviluppati separatamente (per ragioni tecniche, organizzative o temporali)
    \item cooperazione di processi in precedenza separati 
    \item cooperazione (o fusione) di enti o aziende indipendenti 
\end{itemize} ed è stimolata dallo sviluppo delle reti.

Possiamo quindi dividere i livello di federazione su tre categorie tra loro ortogonali (ovvero indipendenti): 
\begin{itemize}
    \item \textbf{Autonomia} che fa riferimento al grado di indipendenza tra i nodi. Si distinguono diverse forme di autonomia
    \begin{itemize}
        \item \textbf{di progetto}: ogni nodo adotta un proprio modello dei dati e sistema di gestione delle transazioni.
        \item \textbf{di condivisione}: ogni nodo sceglie la porzione di dati che intende condividere con altri nodi.
        \item \textbf{di esecuzione}: ogni nodo decide in che modo eseguire le transazioni che gli vengono sottoposte.
    \end{itemize}
    Da qui possiamo distinguere diversi tipi di autonomia
    \begin{itemize}
        \item \textbf{DBMS strettamente integrati}: nessuna autonomia. Dati logicamente centralizzati, esiste quindi un unico data manager responsabile delle transazioni applicative e i data manager locali non operano in modo autonomo
        \item \textbf{DBMS semi-autonomi}: Ogni data manager è autonomo ma partecipa a transazioni globali, una parte dei dati è condivisa. Richiedono modifiche architetturali per poter fare parte della federazione 
        \item \textbf{DBMS totalmente autonomi} (o Peer to Peer) : Ogni DBMS lavora in completa autonomia ed e’ inconsapevole dell’esistenza degli altri
    \end{itemize} 
    \item \textbf{Distribuzione} che fa riferimento alla distribuzione dei dati. Inoltre la distribuzione è di tipologia:
    \begin{itemize}
        \item \textbf{client/server}: la gestione dei dati è concentrata nei server, mentre i client forniscono l’ambiente applicativo e la presentazione. 
        \item\textbf{ peer-to-peer}: non c'è distinzione tra client e server, e tutti i nodi del sistema hanno identiche funzionalità DBMS.
        \item nessuna distribuzione
    \end{itemize}
    \item \textbf{Eterogeneità} riguarda invece:
    \begin{itemize}
        \item \textbf{il modello dei dati} (relazionale, XML, Object Oriented).
        \item \textbf{il linguaggio di query} (diversi dialetti SQL, Query by Example, Linguaggi di interrogazione OO o XML).
        \item \textbf{la gestione delle transazioni} (diversi protocolli per il concurrency control e per il recovery)
        \item \textbf{schema concettuale e schema logico} (un concetto rappresentato in uno schema come attributo e in un altro come entità).
    \end{itemize}
\end{itemize}

Tipologie più rilevanti di DBMS:
\begin{itemize}
  \item \textbf{DBMS distribuito omogeneo (DDBMS)}: quando si ha alta distribuzione ma non si hanno autonomia ed eterogeneità (gestiti solitamente dallo stesso vendor).
  \item \textbf{DBMS distribuiti eterogenei}: strettamente integrati ma con eterogeneità di varia.
  \item \textbf{DBMS federati e logicamente integrati}: semi-autonomi, eterogenei.
  \item \textbf{DBMS federati distribuiti}: quando si ha alta  distribuzione, semi autonomia e non eterogeneità.
  \item \textbf{multi Data Base MS}: totalmente autonomi ed eventualmente omogenei o eterogenei.
  \item \textbf{Data Warehouse}: centralizzati, ma risultato della integrazione fuori linea di fonti eterogenee.
\end{itemize}

\subsection{Architetture dati e funzioni}
I DBMS distribuiti omogenei (DDBMS) sono strettamente integrati, omogenei. Studiamo uno schema in cui si passa da un sistema centralizzato ad un sistema distribuito.\\
Per ciascuna funzione (query processing, transaction manager, ecc) vi può essere una gestione:
\begin{itemize}
    \item Centralizzata/gerarchica vs distribuita
    \item Con assegnazione statica vs dinamica dei ruoli
\end{itemize}
Nei DDBMS gli schemi locali sono viste dello schema globale che perciò viene progettato prima.

La natura fortemente integrata dei DDBMS porta a adottare per essi un approccio top-down alla progettazione, che rispetto alla progettazione di applicazioni DBMS introduce una nuova fase, la progettazione della distribuzione.

\textbf{Analisi dei requisiti $\to$ Progettazione concettuale $\to$ Progettazione logica $\to$ Progettazione fisica.\\}
Inoltre, progettazione logica e fisica sono ora effettuate sui nodi locali.\\
\textbf{Analisi dei requisiti $\to$ Progettazione concettuale $\to$ Progettazione della distribuzione $\to$ Progettazione logica locale $\to$ Progettazione fisica locale.}
 \newline \newline
Si introduce il concetto di \textbf{portabilità}, ovvero la capacità di eseguire le stesse applicazioni DB su ambienti runtime diversi (anche con SQL diversi e differenti dallo standard). 
Si ha anche il concetto di \textbf{interoperabilità} (tra vendors diversi), ovvero la capacità di eseguire applicazioni che coinvolgono contemporaneamente sistemi diversi ed eterogenei (con zero autonomia). A tal fine sono stati introdotti dei \textit{middleware}, tra cui \textbf{ODBC} che si occupa dell'accesso a dati di diversi vendor. ODBC, a livello architetturale, si pone sopra il DBMS e da un'immagine indipendente da ciò che c'è sotto. Si hanno anche dei protocolli, come \textbf{X-Open Distributed Transaction Processing (\textit{DTP})}, che consentono di eseguire delle transazioni secondo una logica diversa.

Normalmente è limitata al livello di accesso.
\begin{itemize}
    \item Basi di Dati Parallele: incrementano le prestazione mediante parallelismo sia di storage devices che di processore
    \item Basi di Dati Replicate: replicano la stessa informazione su diversi server per motivi di performance.
    \item Data warehouses: DBMS centralizzati, risultato della integrazione di fonti eterogenee, dedicati specificamente alla gestione di dati per il supporto alle decisioni.
\end{itemize}


