\section{Lezione del 12 ottobre}
\subsection{Logica di Hoare}

\begin{itemize}
    \item \textbf{Istruzioni di scelta}: L'istruzione di     scelta fa parte delle regole di derivazione viste     nella lezione precedente. 

    L'istruzione si presenta nella forma: $\{p\} \textbf{if} \ B \ \textbf{then} \ C \ \textbf{else} \ D \ \textbf{endif} \{q\}$.
    
    
    Cerchiamo premesse che permettono di derivare questa istruzione. Nella situazione attuale la condizione che possiamo eseguire può essere o \textit{C} o \textit{D}. Valutiamo separatamente le due possibilità:
    \begin{itemize}
        \item Nel caso in cui la condizione del \textbf{if} fosse vera: $\{p \land B\}$ C $\{q\}$ 
            \item All'inizio dell'esecuzione è falsa la condizione \textit{B}, in questo caso: $\{p \land \neg B\}$ D $\{q\}$ 
        \end{itemize}
        Entrambe queste due triple sono delle premesse dalle quali possiamo dedurre la tripla risultante, ricaviamo quindi la formula generale della regola di derivazione
            $\displaystyle \frac{
                    \{p \land B \} C \{q\} \ \{p \land \neg B\} D \{q\}
                }
                {
                    \{p\} \textbf{if} \ B \ \textbf{then} \ C \ \textbf{else} \ D \ \textbf{endif} \{q\}
                }$
                
    \item \textbf{Istruzioni interattive} si presentano nella forma $\textbf{while} \ B \ \textbf{do} \ c \ \textbf{endwhile}$ che chiameremo $W$ successivamente
    abbiamo quindi la seguente tripla.
    $\{p \} W \{q\}$  Se eseguo il comando W a partire dallo stato in cui vale p, alla fine deve valere q.
    
    \begin{itemize}
        \item correttezza parziale, supponiamo a priori che l'esecuzione termini, senza dimostrarlo. Lo supponiamo solamente. La tripla ci darà solamente una soluzione parziale.
        \item si deve dimostrare che l'esecuzione termini. Si procede quindi prima dimostrando la correttezza parziale aggiungendo poi la dimostrazione per l'esecuzione finita.
    \end{itemize}
    Noi studieremo sempre triple che nella post condizione troviamo $\{p\}W\{q \land \neg B\}$, questo perché in ordine per terminare il programma la condizione $B$ deve risultare falsa.\\
    Ipotizziamo che all'inizio la condizione \textbf{B} sia vera, avendo quindi la precondizione $\{i \land B\}$, in questo caso ci troviamo ad eseguire la condizione C, suppongo di poter derivare, tramite C eseguito una sola volta, nuovamente $i$, mi ritrovo la seguente tripla: $\{i \land B\}C\{i\}$.
    Questo significa che, se vale la tripla elencata sopra, la nostra $i$, che resta medesima ad inizio e alla fine dell'esecuzione della condizione C, è una \textbf{\textit{invariante di ciclo}} per C.\\
    Qualora valga dopo la singola esecuzione di $C$ allora avrei ancora $\{i\land B\}$ e dovrei eseguire nuovamente $C$ e dopo varrà ancora $i$ sicuramente e bisogna ristudiare $B$ per capire come procedere, in quanto $i$ resterà vera per qualsiasi numero di iterazioni di $C$. Si ha che $i$ resterà vera, teoricamente, anche una volta usciti dal ciclo. \\
    Qualora non valga $B$ si ha che: $\{i\land \neg B\}W\{i\land \neg B\}$. Questo ci porta a poter dire che  $\displaystyle \frac{\{i \land B\}C\{i\}}{\{i\}W\{i\land \neg B\}}$, che corrisponde alla nostra regola di iterazione. 
\end{itemize}


\subsection{Notazioni}
Una notazione da tenere a mente è $\vdash$ (derivabile) (\textbf{$\vdash \ \{p\}C\{q\}$}), intendiamo dire che quella tripla è stata derivata applicando le regole di derivazione. (sintassi, perché è una notazione puramente formale)

Una seconda notazione è $\models$ (vera) (\textbf{$\models \ \{p\}C\{q\}$}). Possiamo leggerlo nella forma \textit{La tripla è vera se ogni volta che si esegue C in uno stato della memoria che soddisfa p si raggiunge uno stato della memoria in cui è vera q}. (semantico)

 

\subsection{Esercizi}



--------

