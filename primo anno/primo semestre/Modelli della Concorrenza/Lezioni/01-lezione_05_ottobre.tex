\section{Lezione del 5 ottobre}

La concorrenza è una caratteristica dei sistemi di elaborazione nei quali può verificarsi
che un insieme di processi o sotto-processi(thread) computazionali sia in esecuzione nello
stesso istante. 

Per i programmi che presentano concorrenza al loro interno possimao definire delle caratteristiche generali.
\begin{itemize}
    \item Competizione per l'accesso alle risorse condivise
    \item Cooperazione per un fine comune
    \item Coordinamento di attività diverse
    \item Sincronia/Asincoria 
\end{itemize}
%------------------------------------------------

\subsection{Logica proposizionale}
\subsubsection{Sintassi}
Definiamo la sintassi: \hfill \newline
$\mathbf{P} = \{p_1,p_2 \dots, p_i\}$, sono i nostri simboli preposizionali (infiniti ma numerabili) \hfill \newline
$\bot$ e $\top$ sono invece le nostre costanti logiche (anche atomiche) \hfill \newline
$\neg, \ \lor, \ \land , \ \to, \ \Leftrightarrow$ rappresentano i nostri connettivi logici, attraverso loro possiamo creare delle formule ben formate. \hfill \newline
$( \,,) $ sono invece i delimitatori delle componenti atomiche oppure delle formule ben formate.
\bigskip

Vogliamo definire l'insieme delle formule ben formate $F_{Prop}$. L'insieme viene definito in modo induttivo:
\begin{enumerate}
    \item $\top, \bot, p_i \in F_{Prop}$
    \item $A \in F_{Prop},  B \in F_{Prop}$: ovvero siano A e B delle formule ben formate, possiamo allora dire che:
    \begin{itemize}
        \item $(\neg A), (A \lor B), (A \land B), (A \to B), (A \Leftrightarrow B) \in F_{Prop}$, ovvero che unendo le due formule ben formate attraverso dei connettivi logici, il risultato corrisponde a delle formule ben formate. 
    \end{itemize}
\end{enumerate}

\subsubsection{Semantica}
La semantica di una formula è il suo valore di verità, ovvero il fatto che sia vera o falsa. 

Sia $V: P \to \{0, 1\}$ una assegnazione booleana che assegna ad ogni proposizione atomica un valore di verità. Questa assegnazione, in maniera induttiva, si estende alle formule:

Sia $I_v(p_i) = V(p_i), I_v(\top) = 1, I_v(\bot) = 0$, dobbiamo ora definire per ogni forma composta definita, come queste formano il valore di verità della formula composta.
\begin{itemize}
    \item $I_v(\neg A) = 1 - I_v(A)$ (negazione del valore ottenuto)
    \item $I_v(A \lor B) = I_v(A) \lor I_v(B)$
    \item $I_v(A \land B) = I_v(A) \land I_v(B)$
    \item $I_v(A \to B) = I_v(\neg A) \lor I_v(B)$
    \item $I_v(A \Leftrightarrow B) = I_v(\neg A \lor B) \land I_v(\neg B \lor A)$ (qui si potrebbe andare avanti)
\end{itemize}
