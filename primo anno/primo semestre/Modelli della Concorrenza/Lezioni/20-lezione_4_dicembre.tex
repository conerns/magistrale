\section{Logiche temporali e model-checking}
I metodi di model-checking rendono possibile l'esaminazione dei modelli formali per determinare se soddisfano determinate proprietà di nostro interesse. 
\subsection{Correttezza dei programmi concorrenti}
Dato l'esempio Java, dove viene rappresentato un sistema Consumatore Produttore che interagiscono in modo concorrente, gestito tramite Thread, abbiamo compreso che questo non è un caso sempre possibile, non sempre abbiamo un programma che effettua la gestione per noi. Ci sono però dei metodi che ci vengono incontro per capire come garantire, o capire, la correttezza dei programmi concorrenti.\\
Alcune delle proprietà che devono essere garantite da un programma, affinché possiamo affermare la corretta gestione delle concorrenze, sono:
\begin{itemize}
    \item ogni oggetto prodotto viene prima o poi consumato
    \item nessun oggetto viene consumato più di una volta
    \item il sistema non raggiunge mai uno stato di deadlock
    \item due processi non si trovano mai contemporaneamente nella sezione critica
    \item se un processo richiede l’accesso alla sezione critica, prima o poi avrà il permesso. 
\end{itemize} 
Ci sono sistemi per i quali non è possibile applicare la logica di Hoare: si tratta di sistemi concorrenti. \\
La logica di Hoare si può adattare in una certa misura a sistemi concorrenti ma solo per programmi che obbediscono allo schema generale: dati d’ingresso, elaborazione e stato finale. \\
Per sistemi con un comportamento potenzialmente infinito nel tempo bisogna trovare un altro metodo.

\subsection{Sistemi reattivi}
Sono sistemi concorrenti, distribuiti e asincroni, che non obbediscono al paradigma input-computazione-output e pertanto non è possibile analizzarli con gli strumenti della logica di Hoare.
All’interno di un sistema ci possono essere dei componenti sequenziali con un compito limitato nel tempo la cui correttezza può essere dimostrata utilizzando la logica di Hoare.

\subsection{Analisi di sistemi concorrenti}
Problema: stabilire se un sistema reattivo è corretto.\\
Metodo: si esprime il criterio di correttezza come formula di un opportuno linguaggio logico, si rappresenta (modella) il sistema nella forma di \textit{sistemi di transizioni} e si valuta se la formula è vera nel sistema di transizioni.\\
Strumenti: sistemi di transizioni (modelli di Kripke), logiche temporali e algoritmi.

\subsection{Sistemi di transizioni}
Gli elementi essenziali di un sistema di transizioni son gli \textbf{stati} e \textbf{transizioni di stato}.
\[A = (Q, T)\]
dove $Q$ è l’insieme degli stati e $T \subseteq Q \times Q$ è l’insieme delle transizioni di stato. Deduciamo che un sistema di transizioni è finito se l’insieme degli stati è finito.
In un sistema di transizioni, un cammino è una sequenza di stati in cui ogni coppia di stati adiacenti è legata da una transizione: \[\pi = q_0 q_1 \dots \; | \; \forall i, \; (q_i, q_{i+1}) \in f\]
Un cammino massimale è un cammino che non può essere ulteriormente esteso: 
\begin{itemize}
    \item è finito se termina in uno stato dal quale non esce alcuna transizione
    \item è infinito se percorre infinite volte uno o più stati.
\end{itemize}
Il suffisso di ordine $i$ di $\pi$ è il cammino $\pi^{(i)} = q_i q_{i+1} \dots$.

\subsection{Modello di Kripke}


Sia $Z = \{z_1, z_2, \dots \}$ l’insieme di proposizioni atomiche.
Dato un sistema di transizioni $A = (Q, T)$, si associa a ogni stato $q \in Q$, l’insieme delle proposizioni atomiche che sono vere in quello stato: \[I : Q \to 2^{AP}\]
$2^Z$ perché è l’insieme potenza di $AP$ (l’insieme di tutti i sottoinsiemi di $AP$). 

Un modello di Kripke è un’estensione del sistema di transizioni ed è definito come: $A = (Q, T, I)$.

{\centering
\begin{tikzpicture}[node distance=2.3cm,>=stealth',bend angle=45,auto]
  \tikzstyle{place}=[circle,thick,draw=blue!75,fill=blue!20,minimum size=6mm]
  \tikzstyle{transition}=[rectangle,thick,draw=blue!75,minimum size=4mm]

    \node [place] (uno)   [label=left:$q$] {1}; %P1
    \node [place] (due)   [right of=uno,label=above:$q$] {2}
        edge[pre] (uno); 
    \node [place] (tre)   [below of=due,label=right:$q \lor p$] {3}
        edge[pre] (due); 
    \node [place] (quattro)   [below of=uno,label=left:$r\lor p$] {4}
        edge[pre] (tre)
        edge[post] (uno); 
    \node [place] (cinque)   [right of=due,label=right:$p$] {5}
        edge[pre] (due);
    \node [place] (sei)   [below of=cinque,label=right:$p$] {6}
        edge[post, bend right] (cinque)
        edge[pre, bend left] (cinque);
\end{tikzpicture}
\par}
Supponiamo di avere come preposizioni atomiche $AP = \{p,q,r\}$, abbiamo inoltre $Q=\{1,2,3,4,5,6\}$, e l'insieme delle transizioni $T=\{(1,2),(2,3),(2,5),(5,6) \dots\}$, supponiamo inoltre che la funzione di interpretazione abbia una sua definizione: $I(4) = {p,r}$. Questo viene interpretato come \textit{la funzione I sullo stato quattro, ha come valore vero i valori rappresentati in sua vicinanza, in questo caso $r\lor p$, indica che sono vere entrambe}\\
\subsection{Logica temporale lineare (Linear Temporal Logic, LTL)}
\subsubsection{Sintassi}
\begin{itemize}
    \item $AP = \{p_1, p_2, \dots, p_i, q, r,  \dots\}$	proposizioni atomiche\\
        Non facendo riferimento al tempo, possono essere verificate immediatamente in un certo stato.
    \item $FBF_{LTL}$	insieme delle formule ben formate
        \begin{enumerate}
            \item Ogni proposizione atomica è una formula ben formata.
            \item Le costanti logiche \textbf{true} ($\top$) e \textbf{false} ($\bot$) sono formule ben formate.
            \item Induzione strutturale: se $\alpha$ e $\beta$ sono formule ben formate, anche $\neg \alpha$ e $\alpha \lor \beta$ sono formule ben formate. E quindi lo sono anche $\alpha \land \beta$, $\alpha \implies \beta$ e $\alpha \iff \beta$.
            \item Induzione strutturale (per introdurre le formule temporali): se $\alpha$ e $\beta$ sono formule ben formate lo sono inoltre formule formate anche le seguenti notazioni:
            \begin{enumerate}
                \item $X \alpha$, nel prossimo stato (di computazione)
                \item $F \alpha \;(\lozenge \alpha)$, prima o poi (eventually, finally) $\alpha$ diventerà vera
                \item $G \alpha \; (\square \alpha)$, sempre (globally, always), in questo caso $\alpha$ è sempre vera
                \item $\alpha\  \bigcup \beta$, fino a quando (until), 
            \end{enumerate}
            Per $\alpha \bigcup \beta$ la si potrà ritrovare riscritta $\bigcup(\alpha \cdot \beta)$, questo perché $\alpha \bigcup \beta \equiv \bigcup(\alpha \cdot \beta)$\\
            Alcune volte possiamo ritrovare $\lozenge$ per andare a indicare
        \end{enumerate}
\end{itemize}

\subsubsection{Semantica}
Si interpretano le formule di LTL su un modello di Kripke, procedendo in due fasi:
\begin{enumerate}
    \item Si definisce un criterio per stabilire se una formula $\alpha$ è vera in un cammino massimale $\pi$.
    \item Si dice che la formula è vera rispetto a uno stato $q$ del modello di Kripke se è vera in tutti i cammini massimali che partono da $q$.
\end{enumerate}

\subsubsection{Esempio}
{\centering
\begin{tikzpicture}[node distance=2.3cm,>=stealth',bend angle=30,auto]
  \tikzstyle{place}=[circle,thick,draw=blue!75,fill=blue!20,minimum size=6mm]
  \tikzstyle{transition}=[rectangle,thick,draw=blue!75,minimum size=4mm]

    \node [place] (uno)   [label=left:$p$] {$q_1$}; %P1
    \node [place] (due)   [right of=uno,label=right:$p \lor s$] {$q_2$}
        edge[pre] (uno); 
    \node [place] (tre)   [below of=due,label=left:$r \lor s$] {$q_4$}
        edge[loop right] (tre)
        edge [pre] (due)
        edge [post] (uno); 
    \node [place] (quattro)   [below of=uno,label=left:$r\lor p$] {$q_3$}
        edge[pre, bend left] (uno)
        edge[post, bend right] (uno); 
\end{tikzpicture}
\par}

Dal che supponiamo di essere interessati a:
\begin{itemize}
    \item $\alpha$ F $s$
    \item $\beta$ F $r$
    \item $\gamma$ G ($p\lor r$)
    \item $\delta$, $p \bigcup s$ 
\end{itemize}

Quello che abbiamo, per quanto riguarda i cammini, dobbiamo vedere se le nostre formule sono vere nei cammini indicati:
\begin{itemize}
    \item $\Pi_1$: $q_1,\ q_3, \ q_1,\ q_3  \ q_1, \ q_3 \dots$ abbiamo che $\quad \Pi_1 \nvDash \alpha,\Pi_1 \vDash \beta,\Pi_1 \vDash \gamma,\Pi_1 \nvDash \delta$
    \item $\Pi_2$: $q_1,\ q_2, \ q_4,\ q_4  \ q_4, \ q_4 \dots$ abbiamo che $\quad \Pi_2 \nvDash \alpha,\Pi_2 \vDash \beta,\Pi_2 \vDash \gamma,\Pi_2 \nvDash \delta$
    \item $\Pi_3$: $q_1,\ q_2, \ q_4,\ q_3, \ q_1, \ q_3 \dots$ abbiamo che $\quad \Pi_3 \vDash \alpha,\Pi_3 \vDash \beta,\Pi_3 \vDash \gamma,\Pi_3 \vDash \delta$
    \item $\Pi_4$: $q_1,\ q_2, \ q_4,\ q_3, \ q_1, \ q_3 \dots$ abbiamo che $\quad \Pi_4 \vDash \alpha,\Pi_4 \vDash \beta,\Pi_4 \vDash \gamma,\Pi_4 \nvDash \delta$
\end{itemize}