\section{Lezione del 26 ottobre}
LTS (sistemi di transizioni etichettati)

I LTS sono modelli per definire il comportamento di un processo CCS che derivano dal modello degli automi a stati finiti, che però sono usati come riconoscitori di linguaggi. La differenza è che negli LTS non si ha l'obbligo di avere un insieme finito di stati. 
Un LTS è definito da $(S, Act, T, s_0)$, dove:
\begin{itemize}
    \item \textbf{S} l'insieme degli stati
    \item \textbf{Act} (Action), è un insiem di nomi di azioni.
    \item \textbf{T}, sono le transazioni che solitamente vengono specificate o come insieme di triple $T = \{s,a,s'\}$, con $s,s' \in S$ e $a \in Act$, o come una relazione di transizione definiti sul prodotto cartesiano $T \subseteq S \times Act \times S$. Una scrittura differente potrebbe essere $(s,a,s') \in T$, oppure in maniera equivalente avere $s \stackrel{a} \to s'$
    \item $s_o$, è lo stato iniziale $\in S$
\end{itemize}
$\iff (s,a,s') \in T  $, il sistema lo posso rappresentare sia come grafo nel seguente modo: 

{\centering
    
    \begin{tikzpicture}[align=center,node distance=2cm]
    \node[state] (q1) {$s$};
    \node[state, right of=q1] (q2) {$s'$};
    \draw (q1) edge[above] node{a} (q2);
    \end{tikzpicture}
\par
} 

Oppure lo posso rappresentare elencando le diverse relazioni di transizioni. La relazione di transizione può essere estesa a una sequenza di azioni .
$$
s \stackrel{a}\to s' \; \text{ estesa a } \; s \stackrel{w}\to s' \; \text{ con } w \in Act^*
\iff 
$$
\begin{align*}
    \begin{cases} 
        \text{se } w=\epsilon & \text{allora }  s=s' \\
        \text{se } w=a \cdot x  \text{ con } a \in Act, x \in Act^*  & \text{allora } \text{ se } s \to^a s'' \to^x s'
    \end{cases}
\end{align*}

In generale diciamo che abbiamo $s \to s' $ sse $\exists a \in Act : s \stackrel{a} \to s'$, e quindi in qualche modo questa relazione $\quad \to \quad$, non etichettata, è uguale a $\displaystyle \bigcup\limits_{a\in Act}\stackrel{a}{\rightarrow}$. \\
Possiamo fare lo stesso per $s \to^* s'$ sse $\exists w \in Act^* : s \stackrel{w} \to s'$, e quindi posso avere 
$\quad \to^* \quad$, non etichettata, uguale a $\displaystyle \bigcup\limits_{w\in Act^*}\stackrel{w}{\rightarrow}$.\\

La relazione $\rightarrow^*$ è la chiusura riflessiva e transitiva della relazione $\rightarrow$ (tale relazione non è simmetrica), avendo sempre $s\rightarrow^*s'$ ed essendo garantita la transitività.\\

\subsubsection{Piccolo ripasso}
Data una relazione binaria $R$ su $X$. $R \subseteq X \times X$ \begin{itemize}
    \item $R$ è \textbf{riflessiva} $\iff \forall x \in X \quad (x,x) \in R$ ovvero $xRx$ 
    \item $R$ è \textbf{simmetrica} $\iff$ se $(x,y) \in R$ allora $(y,x) \in R \quad \forall x,y \in X$ 
    \item $R$ è \textbf{transitiva} $\iff$ se $(x,y) \in R \land (y,z) \in R$ allora $(x,z) \in R$ 
\end{itemize}
Se $R$ è simmetrica, riflessiva e transitiva, allora è una relazione di equivalenza.
Una \textbf{classe di equivalenze} data un elemento $x \in X$ è l’insieme di tutti gli elementi di $X$ in relazione con esso: $[x] = \{y \in X \; | \; (x,y) \in R\}$. \\
L’insieme $X$ è ripartibile in classi di equivalenza in quanto costituiscono insiemi disgiunti. Infatti, due classi di equivalenza sono o esattamente uguali o la loro intersezione è nulla. L’unione di tutte le classi di equivalenza di $X$ è pertanto uguale a $X$. 

Siano $R, R', R''$ relazioni binarie su $X$, ovvero $R, R', R'' \subseteq X \times X$. $R'$ è la *chiusura riflessiva/simmetrica/transitiva* di $R$ sse: 
\begin{enumerate}
    \item $R \subseteq R'$
    \item $R'$ è riflessiva/simmetrica/transitiva
    \item $R'$ è la più piccola relazione che soddisfa i punti 1 e 2. Dire che è la più piccola relazione corrisponde a dire che $\forall R''$ se $R \subset R'' \land R''$ è riflessiva/simmetrica/transitiva, allora deve succedere che $R' \subseteq R''$
\end{enumerate}

\subsection{CCS}
Per definire il \textbf{Calculus of Communicating Systems (\textit{CCS})} puro (astraendo l'aspetto delle strutture dati etc$\ldots$) dobbiamo definire che abbiamo: 
\begin{itemize}
    \item $K$, ovvero un insieme di nomi di processi, che possono anche essere simboli di un alfabeto
    \item $A$, ovvero un insieme di nomi di azioni, che possono essere azioni di sincronizzazione con l’ambiente o tra componenti o azioni interne di sincronizzazione avvenuta tra due componenti.
    \item $\overline{A}$, ovvero l'insieme di nomi delle \textit{coazioni} contenute in $A$, $\forall\, a\in A \quad \exists \quad \overline{a}\in \overline{A}$, quindi:$\overline{A}=\{\overline{A}|\,a\in A\}$ ovviamente si ha che:$\overline{\overline{a}}=a$
    \item $Act=A\cup \overline{A}\cup \{\tau\}$ dove $\tau\not\in A$ corrisponde all'azione di sincronizzazione tra $a$ e $\overline{a}$. Le due azioni $A$ e $\overline{A}$ sono \textit{azioni osservabili} e lo possiamo indicare con: $\mathcal{L}=A\cup \overline{A}$, mentre $\tau$ non è osservabile.\\ Ricordando che osservare un'azione significa poter interagire con essa.
\end{itemize}
Le azioni osservabili si devono sincronizzare con altre componenti, mentre le azioni non osservabili sono il risultato di una sincronizzazione.
\subsubsection{Processi CSS}
Sono delle espressioni scritte in linguaggio CCS, e un sistema CSS è dato da un insieme di processi $p \in K$, questi processi saranno definiti dalla formula $p = \textnormal{espressione CSS}$ e avrò solo una equazione $\forall \ p \in K$. \\

Dato un sistema CCS con $Act = A \cup \overline{A} \cup \{\tau\}$ e un insieme di processi CCS, gli si va ad assegnare un LTS, che avrà ($S=\text{processi CCS}, Act, T, p_0$), tramite delle regole di inferenza, in cui si ha una premessa (o più), delle condizioni, delle conseguenze (o conclusioni). Dare un significato tramite LTS, regole di inferenza e sintassi è detto
\textbf{semantica operazionale strutturale}.

Un processo CCS può essere:
\begin{itemize}
    \item \textbf{Nil} (0): corrisponde al sistema di transizioni con un unico stato etichettato “nil” senza nessuna transizione.
    \item \textbf{Operazione di prefisso}: $\alpha \cdot p$, dove $\alpha \in Act$ e $p \in \text{Processi CCS}$. La regola di inferenza associata è: \\ $\frac{}{\alpha \cdot p  \stackrel{\alpha}\to p}$
    \item \textbf{Operazione di somma} $p_1 + p_2$, dove $p_1, p_2 \in \text{Processi CSS}$.   Regola di inferenza: $p_1 \stackrel{\alpha}\to p'_1$. Il $+$ corrisponde alla scelta, quindi possiamo definire il comportamento con le regole di inferenza, con $\alpha$ e $\beta \ \in \ Act$ e $p_1 \ p_2 \in Proc_{CSS}$(processi CSS):
    \begin{itemize}
        \item Se sappiamo che $p_1$  può eseguire $\alpha$ e poi comportarsi come $p_1'$, allora $p_1 + p_2$ può eseguire $alpha$. $\frac{p_1 \stackrel{\alpha}\to p_1'}{p_1 + p_2 \stackrel{\alpha}\to p_1'}$
        \item Possiammo inoltre eseguire $\beta$, e seguendo lo stesso ragionamento per $p_1$, ottenendo
         $\frac{p_2 \stackrel{\beta}\to p_2'}{p_1 + p_2 \stackrel{\beta}\to p_2'}$
    \end{itemize}
    \item \textbf{Somma di processi}, posso avere quindi multiple somme, nella forma $\displaystyle \sum_{i\in I}p_i$ con $P_i \in Poc_{CSS}$, in questo caso la regola di inferenza associata è: $\frac{p_j\stackrel{\alpha}{\to}p_j'}{\displaystyle \sum_{i\in I}p_i\stackrel{\alpha}{\to}p_j0},\,j\in I$. \\
    Si possono presentare diversi casi: 
    \begin{itemize}
        \item se $I= \varnothing$ avrò che $\displaystyle \sum_{i\in I}p_i=Nil$
    \end{itemize}
    \item \textbf{Composizione parallela} se $p_1$ e $p_2$ sono due processi gli posso mettere in composizione parallela. Quindi $p_1$, $p_2 \ \in Prco_{CSS}$. Anche qui abbiamo le regole di inferenza
    \begin{itemize}
        \item Se $p_1$ può eseguire $\alpha$ e poi comportarsi come $p_1'$, allora se io ho $p_1 | p_2$, il processo può eseguire $\alpha$ e poi comportarsi come $p_1 | p_2$. Ovvero sarà: $\frac{p_1 \stackrel{\alpha}\to p_1'}{p_1|p_2 \stackrel{\alpha}\to p_1'|p_2}$
        \item Possiamo avere la cosa simmetrica nel caso di $p_2$, risulta che: $\frac{p_2 \stackrel{\alpha}\to p_2'}{p_1|p_2 \stackrel{\alpha}\to p_1|p_2'}$, usiamo in questo caso sempre $\alpha$, ma essa potrebbe essere diversa nella due regole (come valore/concetto).
        \item Può succedere anche che $p_1$ possa eseguire una certa azione $a$, e che $p_2$ sia pronto ad eseguire la sua coazione $\overline{a}$. Questo significa che sono entrambi pronti ad eseguire e possono tutti e due sincronizzarsi.
        $\frac{p_1 \stackrel{a}\to p_1' \land p_2 \stackrel{a}\to p_2'}{p_1|p_2 \stackrel \tau\ \to p_1'|p_2'}$, usiamo $\tau$ nella conseguenza perché vuol dire che i due processi si sincronizzano e $a$ si sincronizza con $\overline{a}$.
    \end{itemize}
    \item \textbf{Operazione di restrizione}, Supponiamo che $L$ sia un sottoinsieme dell'azione $A$, e che $P$ appartenga ai processi $P \in Proc_{CSS}$. Abbiamo quindi $P_{\setminus L}$, significa che  il processo $P$ non può interagire con il suo ambiente con azioni in $L \cup \overline{L}$ , ma le azioni in $L \cup \overline{L}$ sono locali a $P$.
    $\frac{p\stackrel{\alpha}{\to}p'}{p_{\setminus L} \stackrel{\alpha}{\to}p'_{\setminus L}} \quad \alpha,\overline{\alpha}\not\in L,\quad \textnormal{abbiamo che }L\subseteq A$
    
    \item \textbf{Rietichettatura}, $p_{[f]}$, operazione che serve per cambiare il nome ad alcune azioni per poter riutilizzare una specifica di un processo, magari all’interno di un altro sistema dove sono stati usati nomi diversi. Ho quindi una funzione $f$ tale che: $f:Act \to Act$. Ci sono alcune regole che devo garantire:
    \begin{itemize}
        \item $f(\tau) = \tau$
        \item $f(\overline{a}) = \overline{f(a)}$
    \end{itemize}
    Ho quindi $p_{[f]}$ tale che:  $\frac{p \stackrel  {\alpha} \to p'}{p_{[f]} \stackrel{f(\alpha)} \longrightarrow p_{[f]}' }$. Può presentarsi la situazione dove dato un nome di processo $k$, se $k = p$, allora $\frac{p \stackrel  {\alpha} \to p'}{k \stackrel  {\alpha} \to p'} \, \ k = p$. \\
    Quindi dato un CCS posso associare un LTS per quanto riguarda la semantica.
\end{itemize}