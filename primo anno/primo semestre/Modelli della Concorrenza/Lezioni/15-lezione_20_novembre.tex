\section{Lezione del 20 novembre}
Le reti di Petri sono state introdotte da Carl Adam Petri nel 1962 con l’obiettivo di descrivere il flusso di informazione tra le componenti di un sistema complesso.

Nei sistemi distribuiti lo stato globale non è osservabile; quindi la simulazione sequenziale non deterministica (semantica a interleaving) dei sistemi distribuiti è una forzatura, non rappresenta le caratteristiche reali del comportamento. La soluzione di Petri consiste nell’utilizzo delle reti di Petri per i sistemi elementari. In questi se devo specificare il \textbf{produttore-consumatore}, vado a specificare il mio sistema a livello di: Quali sono gli stati locali del sistema e quali sono gli eventi, in questo caso dell'ambiente.

Una transizione, nel modello dei Sistemi di Transizioni Etichettati, prende uno stato globale e lo trasforma in un altro stato globale. Nelle reti di Petri, invece, la transizione è locale che dipende da alcune precondizioni e produce postcondizioni.

Lo \textbf{stato} viene definito da una collezione di stati locali, ovvero condizioni vere. Gli stati locali sono rappresentati da cerchi e rappresentano condizioni booleane (vere se presentano un pallino, false altrimenti).\\

Abbiamo inoltre gli eventi, ovvero le nostre transizione locali, rappresentata con quadrati. Gli stati locali determinano quali eventi sono \textbf{abilitati}, ovvero possono occorrere. \\
Due eventi occorrono in modo \textbf{concorrente} se vengono abilitati contemporaneamente.
\subsection{Le reti elementari}
Una rete $N = (B,E,F)$ è una rete, questa possiamo rappresentarla attraverso un grafo bipartito, tale che:
\begin{itemize}
    \item $B$: insieme finito di condizioni (stati locali, proprietà booleane) rappresentate da cerchi (\textbigcircle).
    \item $E$: insieme finito di eventi (trasformazioni locali di stato, transizioni locali) rappresentati da quadrati ($\square$).
    \item $B \cap E = \varnothing$ e $B \cup E \neq \varnothing$ ($B$ e $E$ sono insiemi disgiunti e non vuoti)
    \item $F \subseteq (B \times E) \cup (E \times B)$: relazione di flusso, rappresentata da una freccia, tale che:\\
        $dom(F) \cup ran(F) = B \cup E$, ovvero non ci sono elementi isolati.
        Una condizione isolata non cambia mai di valore e quindi non è osservabile (in quanto la si osserverebbe solo nel momento in cui dovesse cambiar valore), pertanto non la si modella perché non ha influenza sul sistema. 
        Un evento isolato non ha influenza su alcuna condizione, non ha pertanto né precondizioni né postcondizioni, e non è rilevabile, pertanto non lo si modella. 
\end{itemize}

Sia $x \in B \cup E = X$, possiamo identificare :
\begin{enumerate}
    \item \textbullet $ x = \{y \in X \; | \; (y,x) \in F\}$ come pre-elementi di $x$ (precondizioni o pre-eventi)
    \item $x$\textbullet$= \{y \in X \; | \; (x,y) \in F\}$	come post-elementi di $x$ (postcondizioni o post-eventi)
\end{enumerate}

Il concetto di pre e post lo posso estendere alle sotto-condizioni/eventi. E le possiamo ottenere l'insieme delle precondzioni \textbullet$A$eseguendo eseguendo l'unione di tutti i \textbullet$X$, discorso analogo per le post condizioni. Condizioni ed eventi sono nozioni duali. 

Sia $A \subseteq B \cup E$, allora \textbullet$A = \displaystyle \bigcup_{x \in A} $\textbullet$x$ e $A $\textbullet$ = \displaystyle \bigcup_{x \in A} x$\textbullet\\

La rete $N = (B, E, F)$ descrive la struttura del sistema, ovvero le relazioni tra le condizioni e gli eventi. 
Il comportamento è definito attraverso le nozioni di caso e di regola di scatto o di transizione.\\

Un caso,configurazione oppure stato globale è un insieme di condizioni $c \subset B$ che rappresentano l’insieme di condizioni vere di una certa configurazione del sistema, ovvero un insieme di stati locali che collettivamente individuano lo stato globale del sistema. \
Abbiamo quindi una condizione falsa \textbigcircle  e una condizione vera $\Large \astrosun$\\

Ho una rete e un sottoinsieme di condizioni, che in una certa condizione presumo siano vere. Devo quindi sapere dire quando un evento, data una certa condizione, può occorrere (essere abilitato), per farlo utilizzo una\textbf{ regola di scatto} o \textbf{regola di transizione} \\

Sia $N = (B, E, F)$ una rete elementare e $c \subset B$.
L’evento $e \in E$ è abilitato (può occorrere) in $c$, denotato con $c[e >$ $\iff $\textbullet$ e \subseteq c \; \land \; \{e $\textbullet$\} \cap c = \varnothing$ (quindi se le precondizioni sono vere e le postcondizioni sono false).\\
Se $c$ in $c[e >$ non rappresenta solo le precondizioni dell’evento $e$, ma tutte le condizioni vere in quel momento; quindi le precondizioni di $e$ sono un sottoinsieme di $c$.\\ 
Se $c[e >$, allora, quando $e$ occorre in $c$, genera un nuovo caso $c'$. Ricordando che $c$ rappresentava un insieme, otteniamo che $c'$, in seguito a $c[e > c'$, sarà così composto: $c' = (c \setminus \{$\textbullet$ e\}) \cup \{e $\textbullet$\}$. In poche parole tolgo dall'insieme le pre condizioni e unisco le postcondizioni.\\
Due eventi non interferiscono uno con l’altro (sono indipendenti) se hanno pre e post condizioni disgiunte.

\subsection{Principio di estensionalità}
Principio di estensionalità (il cambiamento di stato è locale): un evento è completamento caratterizzato dai cambiamenti che produce negli stati locali e che sono indipendenti dalla particolare configurazione in cui l’evento occorre.

Sia $N = (B,E,F)$ \textbf{una rete elementare}:
\begin{itemize}
    \item $N$ è semplice $\iff \forall x,y \in B \cup E, \;  $\textbullet$ x = $\textbullet$ y \; \land \; x $\textbullet$ = y $\textbullet$ \implies x = y$\\
        Essendo che le due condizioni $x$ e $y$ cambiano valore di verità in maniera sincronizzata, (altrimenti l’evento per cui sono precondizioni o postcondizioni non si potrà mai verificare), non ha senso separarle in due variabili distinte.
    \item $N$ è pura $\iff \forall e \in E, \; $\textbullet$ e \cap e $\textbullet$ = \varnothing$
        Nel caso in cui \textbullet$ e \cap e $\textbullet$ \neq \varnothing$ (rete non pura), $e$ non occorre mai quindi non viene modellato. 
\end{itemize}
  Nelle reti 1-safe (non elementari), il caso \textbullet$ e \cap e $\textbullet$ \neq \varnothing$ viene visto come un modo più sintetico per rappresentare un’ulteriore condizione che si frammezza tra \textbullet$ e$ e $e $\textbullet.