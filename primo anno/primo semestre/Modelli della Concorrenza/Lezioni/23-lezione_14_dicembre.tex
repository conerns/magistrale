\section{Confronto tra LTL e CTL}
Molte proprietà interessanti si possono esprimere sia in LTL  sia in CTL:
    \begin{itemize}
    \item Invarianti: nella prima parte abbiamo $AG \neg p$ esprimibile come $G \neg p$
    \item Reattività: in questo caso abbiamo  da una parte $AG(p \to AFq)$ e dall'altra $G(p \to Fq)$

\end{itemize}
Abbiamo poi la Reset property, si esprime nella come \textbf{AG EF}$p$,e sta a indicare che da ogni stato raggiungibile in ogni cammino è sempre possibile raggiungere uno stato nel quale vale $p$. Questa proprietà non la si può esprimere in LTL.\\

Abbiamo la formula $FGp$, in CTL, indica che in ogni cammino prima o poi si raggiungerà uno stato a partire dal quale $p$ rimane sempre vera. Questa proprietà non la si può esprimere in LTL.

{\centering
\begin{tikzpicture}[node distance=2.3cm,>=stealth',bend angle=30,auto]
  \tikzstyle{place}=[circle,thick,draw=blue!75,fill=blue!20,minimum size=6mm]
  \tikzstyle{transition}=[rectangle,thick,draw=blue!75,minimum size=4mm]

    \node [place] (uno)   [label=below:$\{p\}$] {$s_0$}
        edge [loop above] (uno); %P1
    
    \node [place] (due)   [right of=uno,label=below:$\{ \ \}$] {$s_1$}
        edge[pre] (uno); 
    \node [place] (tre)   [right of=due,label=below:$\{p\}$] {$s_2$}
        edge [pre] (due)
        edge [loop above] (tre); %P1
\end{tikzpicture}
\par}

Considerato il modello di Kripke sopra, analizziamo la validità di $FG_p$. Andiamo a considerare i cammini massimali
\begin{itemize}
    \item $(s_0)^w$, in questo caso $FG_p$ è verificata
     \item $(s_0 \ s_1) s_2^w$, in questo caso $FG_p$ è verificata
\end{itemize}
Quindi si verifica sempre. Nonostante questo non possiamo trasformala in LTL. Questo perché una possibile traduzione sarebbe AFAG$p$, ma come possiamo notare la \textbf{A} non rispetto la validità della formula. Si trae che:
\[M,s_0 \vDash FGp \quad M,s_0 \nvDash AFAGp\]

Confrontiamo però la formula AFEGp, questo perché comunque io avanzi nell'albero, da qualsiasi punto esiste un cammino in cui $p$ è vera. \textbf{MA} sia:

{\centering
\begin{tikzpicture}[node distance=2.3cm,>=stealth',bend angle=30,auto]
  \tikzstyle{place}=[circle,thick,draw=blue!75,fill=blue!20,minimum size=6mm]
  \tikzstyle{transition}=[rectangle,thick,draw=blue!75,minimum size=4mm]

    \node [place] (uno)   [label=below:$\{ \ \}$] {$s_1$};
    \node [place] (tre)   [right of=uno,label=below:$\{p\}$] {$s_2$}
        edge [pre, bend left] (uno)
        edge [post, bend right] (uno)
        edge [loop above] (tre); %P1
\end{tikzpicture}
\par}
Ci ritroviamo comunque a dire che
\[M,s_1 \nvDash FGp \quad M,s_1 \vDash AFEGp\]

Perché abbiamo un cammino che non rende mai vera $Gp$, è vera però AFAGp. Infatti prima o poi ci sarà un cammino in cui vale \textit{p}.
\subsection{CTL$^*$}
Esiste una logica CTL*, questa estende sia LTL sia CTL, mantenendo i due quantificatori sui cammini, ma eliminando il vincolo di CTL.\\
\textbf{Esempio}: AFG$p \lor$ AFEG $q$ espressa EFG$q$\\

In linea di massima, estendere la capacità espressiva di una logica si paga con un maggior costo computazionale degli algoritmi di model-checking per le formule di quella logica.

\subsection{Equivalenza di modelli rispetto a una logica}
Due modelli di Kripke, $M_1$ e $M_2$, con stati iniziali $q_0$ e $s_0$ si dicono \textit{equivalenti} rispetto a una logica \textbf{L} se, per ogni formula $\alpha \in FBF_L$: \[M_1, q_o \vDash \alpha \iff M_2, s_0 \vDash \alpha\]

\subsection{Insiemi parzialmente ordinati}
Relazione d’ordine parziale su $A$: $\leq$ $\subseteq$ $A \times A$
\begin{enumerate}
    \item \textbf{riflessiva}: $x \leq x$, $\forall x \in A$
    \item \textbf{antisimmetrica}: $(x \leq y \land y \leq x) \implies x = y$, $\forall x, y \in A$
    \item \textbf{transitiva}: $(x \leq y \land y \leq z) \implies x \leq z$, $\forall x,y,z \in A$
\end{enumerate}

Sia $(A, \leq)$ un insieme parzialmente ordinato e $B \subseteq A$. Allora diciamo che $x \in A$ è un \textbf{maggiorante} di $B$ se $y \leq x$, $\forall y \in B$. Mentre $x \in A$ è un \textbf{minorante} di $B$ se $x \leq y$, $\forall y \in B$.\\

Si indica con $B^*$ l’insieme dei maggioranti di $B$ e con $B_*$ l’insieme dei minoranti di $B$.\\
$B$ si dice:
\begin{enumerate}
    \item limitato \textbf{superiormente} se $B^* \neq \emptyset$. 
    \item limitato \textbf{inferiormente} se $B_* \neq \emptyset$.
\end{enumerate}

Per quanto riguarda \textbf{$x \in B$}, questo:
\begin{itemize}
    \item è il \textbf{minimo} di $B$ se $x \leq y$, $\forall y \in B$.
    \item è il \textbf{massimo} di $B$ se $y \leq x$, $\forall y \in B$.
    \item è \textbf{minimale} in $B$ se $y \leq x \implies y = x$.
    \item è \textbf{massimale} in $B$ se $x \leq y \implies y = x$.
\end{itemize}
 
Se $x$ è il minimo di $B^*$, si dice che $x$ è l’estremo superiore (join) di $B$ e si scrive $x = \sup B$ o $x = \bigvee B$. Se $x$ è il massimo di $B_*$, si dice che $x$ è l’estremo inferiore (meet) di $B$ e si scrive $x = \inf B$ o $x = \bigwedge B$. In particolare, se $B = \{x,y\}$, si scrive $x \bigvee y$ per indicare $\bigvee B$ se esiste e $x \bigwedge y$ per $\bigwedge B$ se esiste.\\

Si consideri la logica proposizionale.\\

La nozione di implicazione ha due aspetti: operazione che date due formule costruisce una nuova formula e relazione binaria tra formule.\\
L’implicazione è riflessiva perché qualunque formula $\alpha$ implica se stessa ed è transitiva perché se una formula $\alpha$ implica $\beta$ e la formula $\beta$ implica $\gamma$ allora anche $\alpha$ implicherà $\gamma$.\\

Per quanto riguarda l’antisimmetria, date due formule diverse $\alpha$ e $\beta$ è possibile che $\alpha$ implichi $\beta$ e viceversa. (Esempio: $\alpha \lor \beta \implies \beta \lor \alpha$ e viceversa)\\

Quindi l’implicazione non è una relazione di ordine parziale sull’insieme di tutte le formule bene formate. Si tratta di una relazione di *preordine*, in quanto è riflessiva e transitiva.\\

È possibile trasformare qualunque relazione di preordine in una relazione di ordine parziale su un insieme derivato.
Definita una relazione di equivalenza tra formule, si possono costruire le classi di equivalenza, ovvero insiemi di formule tali che in ogni insieme ci sono formule due a due equivalenti.\\

Con $[FBF_{LP}]_{\equiv}$ si indica l’insieme di tutte le classi di equivalenza, sul quale si può definire una relazione $\implies$ definita nel seguente modo: una certa classe di equivalenza implica un’altra classe di equivalenza se una qualunque delle formule della prima classe implica una qualunque delle formule della seconda classe.\\
$([FBF_{LP}]_{\equiv}, \implies)$ è una relazione di ordine parziale.

\subsection{Reticolo}
Un \textit{reticolo} è un insieme parzialmente ordinato $(L, \leq)$, tale che $\forall x, y \in L$, esistono $x \bigvee y$ (join) e $x \bigwedge y$ (meet).
Un reticolo si dice completo se $\bigvee B$ e $\bigwedge B$ esistono $\forall B \subseteq L$.